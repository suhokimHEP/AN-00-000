\clearpage
\section{Event Selection, Signal and Control Regions}\label{sec:selections}

The signal process of the analysis contains SM $\tau$ fermions for its final state. 
In order to exploit the leptonic decay of $\tau$ lepton, specifically with muon final state for clean signal, the B-Parking triggers are used. 
CMS implemented the B-Parking trigger starting in 2018 of Run 2 for research of lepton universalities. 
For research of R(K$^{*}$,D$^{*}$), muonic final state of B mesons are desired. 
B trigger requires a soft muon with modest displacement (impact parameter) from primary vertex due to b-quark's long lifetime.
B Parking HLT requires a muon with 7-12 GeV with IP 3-6.
pp collisions in LHC produce extremely enormous amount of data, which could be triggered by paths above. 
Current CPU capacity of CMS is limited and not capable of reconstructing the entire event at such high rate at HLT level.
CMS scouts events, which passed L1 trigger, and writes them to a temporary dataset. Later, full HLT and RECO schemes are implemented and served as a B-Parking dataset. 
The prescale factor for HLT is 5-6.


\subsection{Global Tags}
\begin{table}[htb]
\caption{Data and MC Global tags used 2018}
\begin{center}
\begin{tabular}{r|l}\hline
 Data 2018 & 106X\_dataRun2\_v29 \\
 \hline
 MC 2018   & 106X\_upgrade2018\_realistic\_v11\_L1v1 \\
 \hline
\end{tabular}
\label{tab:GT}
\end{center}
\end{table}


\subsection{Trigger Strategy}
We utilize the B-Parking triggers collecting data at L1 and HLT for 2018.
The HLT paths of these triggers are listed in Tables~\ref{tab:triggers18}.
We observe that the triggers become efficient
around the nominal trigger thresholds.
%The standard trigger scale factors are applied in order to correct the MC.

\begin{table}[htb]
\caption{HLT trigger paths used in the analysis 2018.}
\begin{center}
\begin{tabular}{r|l}\hline
\hline
 Data sample & Trigger \\
\hline
 ParkingBPH*-Run2018A & HLT\_Mu9\_IP6\_part* \\
 \hline
 ParkingBPH*-Run2018B & HLT\_Mu12\_IP6\_part* \\
 ParkingBPH*-Run2018C & HLT\_Mu12\_IP6\_part* \\
 ParkingBPH*-Run2018D & HLT\_Mu12\_IP6\_part* \\
 \hline
 \hline
\end{tabular}
\label{tab:triggers18}
\end{center}
\end{table}


% \begin{figure}[h!]
%   \caption{Trigger turn-on curves for the 17 MuonEG HLT path. Leading \pt
%   lepton (left) and subleading \pt lepton (right).}
%   \label{fig:17_trigger_turnon_mueg}
%   \centering
%   \includegraphics[width=0.40\linewidth]{figs/2017_TTOCEMu_ElePt.png}
%   \includegraphics[width=0.40\linewidth]{figs/2017_TTOCEMu_MuPt.png}
% \end{figure}
% \begin{figure}[h!]
%   \caption{Trigger turn-on curves for the 16 DoubleElectron HLT paths. Leading \pt
%   lepton (left) and subleading \pt lepton (right).}
%   \label{fig:16_trigger_turnon_dielectron}
%   \centering
%   \includegraphics[width=0.40\linewidth]{figs/2016_TTOCEle1Pt.png}
%   \includegraphics[width=0.40\linewidth]{figs/2016_TTOCEle2Pt.png}
% \end{figure}
%
% \begin{figure}[h!]
%   \caption{Trigger Efficiency for the 16 DoubleElectron HLT paths. Leading and subleading
%   lepton vs \pt (left) and $\eta$ (right).}
%   \label{fig:16_trigger_turnon__twoD_dielectron}
%   \centering
%   \includegraphics[width=0.40\linewidth]{figs/2016_TTOC_Ele23Ele12_DElePt.png}
%   \includegraphics[width=0.40\linewidth]{figs/2016_TTOC_Ele23Ele12_DEleEta.png}
% \end{figure}
%
% \begin{figure}[h!]
%   \caption{Trigger turn-on curves for the 16 DoubleMuon HLT paths. Leading \pt
%   lepton (left) and subleading \pt lepton (right).}
%   \label{fig:16_trigger_turnon_dimuon}
%   \centering
%   \includegraphics[width=0.40\linewidth]{figs/2016_TTOCMu1Pt.png}
%   \includegraphics[width=0.40\linewidth]{figs/2016_TTOCMu2Pt.png}
% \end{figure}
%
% \begin{figure}[h!]
%   \caption{Trigger Efficiency for the 16 DoubleMuon HLT paths. Leading  and subleading
%   lepton vs \pt (left) and $\eta$ (right).}
%   \label{fig:16_trigger_turnon__twoD_dimuon}
%   \centering
%   \includegraphics[width=0.40\linewidth]{figs/2016_TTOC_Mu17Mu8_DMuPt.png}
%   \includegraphics[width=0.40\linewidth]{figs/2016_TTOC_Mu17Mu8_DMuEta.png}
% \end{figure}
%
%
% \begin{figure}[h!]
%   \caption{Trigger turn-on curves for the 16 MuonEG HLT path. Leading \pt
%   lepton (left) and subleading \pt lepton (right).}
%   \label{fig:16_trigger_turnon_mueg}
%   \centering
%   \includegraphics[width=0.40\linewidth]{figs/2016_TTOCEMu_ElePt.png}
%   \includegraphics[width=0.40\linewidth]{figs/2016_TTOCEMu_MuPt.png}
% \end{figure}
%
% \begin{figure}[h!]
%   \caption{Trigger turn-on curves for the 17 DoubleElectron and DoubleMuon HLT paths. Electron \pt
%   (left) and Muon \pt lepton (right).}
%   \label{fig:17_trigger_turnon_dielectron_and_dimuon}
%   \centering
%   \includegraphics[width=0.40\linewidth]{figs/2017_TTOCEle1Pt.png}
%   \includegraphics[width=0.40\linewidth]{figs/2017_TTOCMu1Pt.png}
% \end{figure}
%
% \begin{figure}[h!]
%   \caption{Trigger Efficiency for the 17 DoubleElectron HLT paths. Leading and subleading
%   lepton vs \pt (left) and $\eta$ (right).}
%   \label{fig:17_trigger_turnon__twoD_dielectron}
%   \centering
%   \includegraphics[width=0.40\linewidth]{figs/2017_TTOC_Ele23Ele12_DElePt.png}
%   \includegraphics[width=0.40\linewidth]{figs/2017_TTOC_Ele23Ele12_DEleEta.png}
% \end{figure}
%
%
% \begin{figure}[h!]
%   \caption{Trigger Efficiency for the 17 DoubleMuon HLT paths. Leading  and subleading
%   lepton vs \pt (left) and $\eta$ (right).}
%   \label{fig:17_trigger_turnon__twoD_dimuon}
%   \centering
%   \includegraphics[width=0.40\linewidth]{figs/2017_TTOC_Mu17Mu8_DMuPt.png}
%   \includegraphics[width=0.40\linewidth]{figs/2017_TTOC_Mu17Mu8_DMuEta.png}
% \end{figure}
%
%
% \begin{figure}[h!]
%   \caption{Trigger turn-on curves for the 17 MuonEG HLT path. Leading \pt
%   lepton (left) and subleading \pt lepton (right).}
%   \label{fig:17_trigger_turnon_mueg}
%   \centering
%   \includegraphics[width=0.40\linewidth]{figs/2017_TTOCEMu_ElePt.png}
%   \includegraphics[width=0.40\linewidth]{figs/2017_TTOCEMu_MuPt.png}
% \end{figure}

\clearpage
\subsection{Search Regions}\label{sec:searhregion}

\begin{itemize}
  \item $\geq$ 1 good primary vertex
  \item One \dilepton pair with 70 GeV $<$ m(\dilepton) $<$ 110 GeV and \pt(\dilepton) $>$ 100 GeV
  \item No additional leptons with \pt $\geq$ 15 GeV
  \item $\geq$ 1 jet
\end{itemize}

Where $\ell = e, \mu$. We refer to the subsets of the search region with
$\ell = e$ and  $\ell = \mu$ as \twoelezh and \twomuzh, respectively.

The leading lepton in the opposite-sign same-flavor (OSSF) pair is required to have \pt $\geq$ 25 GeV,
while the subleading lepton is required to have \pt $\geq$ 15 GeV.
These cuts were chosen to be near the plateau of the trigger efficiency.

The \pt cut on the OSSF lepton pair was optimized for this analysis.
It is well known from standard model searches for associated Higgs production that the Z spectrum
in associated production is harder than that of background. Figures~\ref{fig:zpt}-\ref{fig:zpt2} show
the di-lepton transverse momentum distribution.
Figure \ref{fig:ptossf} shows this with unit normalized \pt(\dilepton) distributions of the total background
and an example signal according to simulation.
We find that a \pt(\dilepton) threshold of approximately 100~\GeV reaches a near
 optimal sensitivity, thus we select a \pt(\dilepton) $>$ 100~\GeV to define the
  search region ($\mathrm{high-\pt}$).
This was found to be true for all scalar masses within the 10-100~\mm lifetime range.
This is shown for an example signal in Figure \ref{fig:ptossfopt}. Sec.~\ref{sec:signaleff}
has more details on the signal efficiency cut-flow yields in the different search
 and control regions.


\begin{figure}[h!]
  \caption{\pt(\dilepton) distributions of the total background from MC and data in 2016.
  (Left) the distribution for the $\mu^+\mu^-$ and (right) $e^+e^-$ channels.}
  \label{fig:zpt}
  \centering
  \includegraphics[width=0.45\linewidth]{figs/v6/2016_TwoMuOffZ_AOD_dileptonNewB_Pt_GH.pdf}
  \includegraphics[width=0.45\linewidth]{figs/v6/2016_TwoEleOffZ_AOD_dileptonNewB_Pt_GH.pdf}
\end{figure}
\begin{figure}[h!]
  \caption{\pt(\dilepton) distributions of the total background from MC and data in 2017.
  (Left) the distribution for the $\mu^+\mu^-$ and (right) $e^+e^-$ channels.}
  \label{fig:zpt2017}
  \centering
  \includegraphics[width=0.45\linewidth]{figs/v6/2017_TwoMuOffZ_AOD_dileptonNewB_Pt.pdf}
  \includegraphics[width=0.45\linewidth]{figs/v6/2017_TwoEleOffZ_AOD_dileptonNewB_Pt.pdf}
\end{figure}
\begin{figure}[h!]
  \caption{\pt(\dilepton) distributions of the total background from MC and data in 2018.
  (Left) the distribution for the $\mu^+\mu^-$ and (right) $e^+e^-$ channels.}
  \label{fig:zpt2}
  \centering
  \includegraphics[width=0.45\linewidth]{figs/v10/2018_TwoMuOffZ_AOD_dileptonNewB_Pt.pdf}
  \includegraphics[width=0.45\linewidth]{figs/v10/2018_TwoEleOffZ_AOD_dileptonNewB_Pt.pdf}
\end{figure}
\begin{figure}[h!]
  \caption{\pt(\dilepton) distributions of the total background from MC and data corresponding to the target
  luminosity used in the analysis and combine ee and $\mu\mu$-channels. }
  \label{fig:zpt2}
  \centering
  %\includegraphics[width=0.45\linewidth]{figs/v10/TwoMuOffZ_AOD_dileptonNewB_Pt.pdf}
  \includegraphics[width=0.45\linewidth]{figs/v10/eemumu_dileptonNewB_Pt.png}
\end{figure}



\begin{figure}[h!]
  \caption{Unit normalized \pt(\dilepton) distributions of the total background and an example signal in simulation.
    The event selection is that of the \twomuzh search region without the  \pt(\dilepton) cut
    with an additional requirement of at least one displaced-jet tag.
    The example signal has a scalar mass of 40 GeV and a proper-lifetime of 10 mm.
  }
  \label{fig:ptossf}
  \centering
  \includegraphics[width=0.60\linewidth]{figs/ptossf.pdf}
\end{figure}


\begin{figure}[h!]
  \caption{Sensitivity quantified by $S$/$\sqrt{S+B}$ as a function of \pt(\dilepton) threshold for one example signal.
    The event selection is that of the \twomuzh search region without the  \pt(\dilepton) cut
    with an additional requirement of at least one displaced-jet tag.
    The example signal has a scalar mass of 40 GeV and a lifetime of 10 mm, but the result does not vary strongly with either.
}
  \label{fig:ptossfopt}
  \centering
  \includegraphics[width=0.60\linewidth]{figs/ptossfopt.pdf}
\end{figure}



The MC expected distribution of the displaced-jet tag multiplicity (\NTAGS)
for SM background and signals, in the \twollzh search region, are shown in
Figure~\ref{fig:ntag_cp_2020_2}.
\NTAGS depends on the scalar mass due to the merging of its decay products
(see right panel of Figure~\ref{fig:scalarpt}).
Therefore, we expect this search to have different sensitivities for different scalar mass
scenarios.
%to peform a generic search for a range of scalar masses,
%we perform the search in the tag multiplicity distribution.

% \begin{figure}[h!]
%   \caption{\NTAGS distributions in the \twoelezh search region (left) and \twomuzh search region (right).}
%   \label{fig:zhntag}
%   \centering
%   \includegraphics[width=0.47\linewidth]{figs/TwoEleZH_nSelectedAODCaloJetTag_log.pdf}
%   \includegraphics[width=0.47\linewidth]{figs/TwoMuZH_nSelectedAODCaloJetTag_log.pdf}
% \end{figure}
\begin{figure}[h!]
  \caption{\NTAGS distribution in the \twollzh search region (left) and \twolldy
  control region (right). Here the BR($H\rightarrow SS \rightarrow bbbb$) is assumed to be 20\% for the signal.}
  \label{fig:ntag_cp_2020_2}
  \centering
  \includegraphics[width=0.47\linewidth,valign=t]{figs/v12/ZH_nSelectedAODCaloJetTag_log.png}
  \includegraphics[width=0.47\linewidth,valign=t]{figs/v12/DY_nSelectedAODCaloJetTag_log.png}

\end{figure}


As can be seen in Figure~\ref{fig:ntag_cp_2020_2}, the largest background is Z bosons decaying to leptons,
 that is \DY and $Z\gamma$ events -- with the former clearly dominating.
The second largest background is events with top quarks, both $t\bar{t}$ and
 single top, hereafter referred to as top background.
The single top background is dominated by the $tW$ channel as these events can
produce an OSSF pair passing the search region cuts when the two W bosons
present in these events (the W from the hard interaction, as well as the one
from the top-quark) decay leptonically. The top background is dominated by $t\bar{t}$ production.
The other backgrounds are a small fraction of the total.
In the 1-tag bin, for example, the Z background is roughly 93\% of the total,
the top backgrounds is roughly 5\% of the total, and the other backgrounds are roughly 2\% of the total.
In the 2-tag bin, the MC statistics are limited and yields have an uncertainty
of about 50~\%, but assuming the central values, the background composition is very similar to that of the 1-tag bin.

\subsection{Control Regions}\label{sec:controlregion}

We use control regions to perform data-driven estimates of the two largest SM
backgrounds -- the \Z and top backgrounds.
Here we define the control regions dedicated to estimate these two SM backgrounds.
The rest of the backgrounds ($\sim$3~\%), hereafter referred to as
\textbf{other backgrounds}, will be taken directly from MC simulation.

\subsubsection{\Z Control Regions}

The control region dominated by the Z background is formed
by identical requirements as the search region with the exception of an \textbf{inverted} \ptossf cut.
Similar to the search regions, the event level requirement for the control region are::
\begin{itemize}
  \item $\geq$ 1 good primary vertex
  \item One \dilepton pair with 70 GeV $<$ m(\dilepton) $<$ 110 GeV and \pt(\dilepton) $<$ 100 GeV
  \item No additional leptons with \pt $\geq$ 15 GeV
  \item $\geq$ 1 jet
\end{itemize}

Where $\ell = e, \mu$. We refer to the subsets of the control region with
$\ell = e$ and $\ell = \mu$ as \twoeledy and \twomudy, respectively.
%We refer to the control region with $\ell = e$ as \textbf{\twoeledy}
%and the control region with $\ell = \mu$ as \textbf{\twomudy}.

The MC expected distribution and the observed data for the displaced-jet
 tag multiplicity (\NTAGS) distribution in the \twolldy control region is shown in Figure~\ref{fig:ntag_cp_2020_2} 
(right).
Contamination in the control region from processes other than the \Z backgound are
accounted for in the global fit signal extraction procedure as discussed in Section~\ref{sec:estimate}.
We note that the signal contribution in these regions is small,
and therefore we do not blind this region in data.

%Don't think we need this one anymore
%\begin{figure}[h!]
%  \caption{\NTAGS distributions in the \twoeledy control region (left) and \twomudy control region (right).}
%  \label{fig:dyntag}
%  \centering
%  \includegraphics[width=0.47\linewidth]{figs/TwoEleDY_nSelectedAODCaloJetTag_log.pdf}
%  \includegraphics[width=0.47\linewidth]{figs/TwoMuDY_nSelectedAODCaloJetTag_log.pdf}
%\end{figure}



\subsubsection{Top Control Regions}

Control regions that are dominated by the $t\bar{t}$ and single-top (top) backgrounds are formed
by changing the OSSF requirement to an opposite-sign different-flavor (OSDF) requirement.
Concretely, we require \emu where $(\ell,\ell')$ = $(e,\mu)$ or $(\mu,e)$.
We define two top-dominated control regions -- one in the high dilepton \pt range of the search region,
and the other in the low dilepton \pt range of the Z background control regions.

%The high dilepton \pt top control region cuts are therefore:
%\begin{itemize}
%  \item $\geq$ 1 good primary vertex
%  \item One \emu pair with 70 GeV $<$ m(\emu) $<$ 110 GeV and \pt(\emu) $>$ 100 GeV
%  \item No additional leptons with \pt $\geq$ 15 GeV
%  \item $\geq$ 1 jet
%\end{itemize}
%We refer to this control region as \textbf{\elemu}.
%
%The low dilepton \pt top control region cuts are:
%\begin{itemize}
%  \item $\geq$ 1 good primary vertex
%  \item One \emu pair with 70 GeV $<$ m(\emu) $<$ 110 GeV and \pt(\emu) $<$ 100 GeV
%  \item No additional leptons with \pt $\geq$ 15 GeV
%  \item $\geq$ 1 jet
%\end{itemize}
The low dilepton \pt top control region cuts are:
\begin{itemize}
  \item $\geq$ 1 good primary vertex
  \item No additional leptons with \pt $\geq$ 15 GeV
  \item $\geq$ 1 jet
\end{itemize}
We refer to this control region as \textbf{\elemuall}.
Figure~\ref{fig:zptelmu} shows the dilepton \pt distributions in the combined \elemuall
region.

\begin{figure}[h!]
  \caption{\elemuall \pt(\dilepton) distributions of the total background from MC and data
  in the (left) 2016 and (right) 2017 data-taking periods.}
  \label{fig:zptelmu}
  \centering
  \includegraphics[width=0.45\linewidth]{figs/v6/2016_EleMuOSOFCombo_AOD_OSOFdileptonNewB_Pt_GH.pdf}
  \includegraphics[width=0.45\linewidth]{figs/v6/2017_EleMuOSOFCombo_AOD_OSOFdileptonNewB_Pt.pdf}
\end{figure}

\begin{figure}[h!]
  \caption{\elemuall \pt(\dilepton) distribution of the total background from MC and data
  in the 2018 data-taking period.}
  \label{fig:zptelmu}
  \centering
  \includegraphics[width=0.45\linewidth]{figs/v6/2018_EleMuOSOFCombo_AOD_OSOFdileptonNewB_Pt.pdf}
\end{figure}

The MC expected distribution of the displaced-jet tag multiplicity (\NTAGS)
for SM background, in the \elemuall control region,
is shown in Figure~\ref{fig:elemuntag_2}.
Contamination in the control region from processes other than the top backgound are
accounted for  in the global fit signal extraction procedure as discussed in
Section~\ref{sec:estimate}. We note that no signal is expected in this region
due to the OSDF lepton pair requirement.


\begin{figure}[h!]
  \caption{\NTAGS distributions in the \elemuall control region.}
  \label{fig:elemuntag_2}
  \centering
  \includegraphics[width=0.47\linewidth]{figs/v12/EleMuOSOFCombo_nSelectedAODCaloJetTag_log.png}
\end{figure}

% \newpage
% \subsubsection{2017}
% Not sure how we plan to add 2017. Make its own section or add to the end of each section of 2016. So,
%  just putting these here for now.
% \begin{figure}[h!]
%   \caption{Z$_{Pt}$ and Dilepton invariant mass distributions with Drell-Yan event selction}
%   \label{fig:Z_2017}
%   \centering
%   \includegraphics[width=0.47\linewidth]{figs/TwoMuDY_AOD_dilepton_Pt2017.png}
%   \includegraphics[width=0.47\linewidth]{figs/TwoMuDY_AOD_dilepton_Mass2017.png}
% \end{figure}
%
% \begin{figure}[h!]
%   \caption{\NTAGS distributions in the \twoelezh search region (left) and \twomuzh search region (right).}
%   \label{fig:zhntag_2017}
%   \centering
%   \includegraphics[width=0.47\linewidth]{figs/TwoEleZH_nSelectedAODCaloJetTag_log2017.png}
%   \includegraphics[width=0.47\linewidth]{figs/TwoMuZH_nSelectedAODCaloJetTag_log2017.png}
% \end{figure}
% \begin{figure}[h!]
%   \caption{\NTAGS distributions in the \twoeledy control region (left) and \twomudy control region (right).}
%   \label{fig:dyntag_2017}
%   \centering
%   \includegraphics[width=0.47\linewidth]{figs/TwoEleDY_nSelectedAODCaloJetTag_log2017.png}
%   \includegraphics[width=0.47\linewidth]{figs/TwoMuDY_nSelectedAODCaloJetTag_log2017.png}
% \end{figure}
% \begin{figure}[h!]
%   \caption{\NTAGS distributions in the \elemu control region (left) and \elemul control region (right).}
%   \label{fig:elemuntag_2017}
%   \centering
%   \includegraphics[width=0.47\linewidth]{figs/EleMuOSOF_nSelectedAODCaloJetTag_log2017.png}
%   \includegraphics[width=0.47\linewidth]{figs/EleMuOSOFL_nSelectedAODCaloJetTag_log2017.png}
% \end{figure}
%
%
%
% \begin{figure}[h!]
%   \caption{Left figure shows tagging variable before the shift is applied and the right figure shows the same variable
%            post-shifting.}
%   \label{fig:tag_correction_IP_2017}
%   \centering
%   \includegraphics[width=0.47\linewidth]{figs/TwoMuDY_AllJets_AODCaloJetMedianLog10IPSig_2017PreShift.pdf}
%   \includegraphics[width=0.47\linewidth]{figs/TwoMuDY_AllJets_AODCaloJetMedianLog10IPSig2017.pdf}
% \end{figure}
%
% \begin{figure}[h!]
%   \caption{Left figure shows...}
%   \label{fig:tag_correction_TA_2017}
%   \centering
%   \includegraphics[width=0.47\linewidth]{figs/TwoMuDY_AllJets_AODCaloJetMedianLog10TrackAngle_2017PreShift.pdf}
%   \includegraphics[width=0.47\linewidth]{figs/TwoMuDY_AllJets_AODCaloJetMedianLog10TrackAngle2017.pdf}
% \end{figure}
%
% \begin{figure}[h!]
%   \caption{Left figure shows...}
%   \label{fig:tag_correction_AM_2017}
%   \centering
%   \includegraphics[width=0.47\linewidth]{figs/TwoMuDY_AllJets_AODCaloJetAlphaMax_2017PreShift.pdf}
%   \includegraphics[width=0.47\linewidth]{figs/TwoMuDY_AllJets_AODCaloJetAlphaMax2017.pdf}
% \end{figure}
