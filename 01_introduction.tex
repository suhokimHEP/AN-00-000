\section{Introduction}\label{sec:introduction}

Discovery of particles at the electroweak scale, such as the top quark at Fermilab's CDF and D0~\cite{topD0,topCDF} and the Higgs Boson at Large Hadron Collider (LHC) in CERN~\cite{higgscms,higgsatlas}, led to discovery of all constituents in the standard model (SM). 
SM helps to describe the nature of fundamental particles and their interactions with precision. 
In spite of its success, the SM suffers from a few obstacles. 
The evidence of neutrino mass and mixing~\cite{neutrino}, the observation of bullet clusters confirming the presence of dark matter (DM)~\cite{Baumgart:2009tn,Kaplan:2009ag,Chan:2011aa,Dienes:2011ja,Dienes:2012yz}, baryon asymmetry~\cite{Cui:2014twa} remain unsolved in the framework of SM. 
In addition, SM suffers from the naturalness problem. 
To solve all such issues, one needs to look for Beyond Standard Model (BSM).

The naturalness problem stems from that the Higgs in the SM is a scalar particle. 
Unlike fermions or gauge bosons, its mass is not protected by any symmetry and subject to large radiative corrections, especially from top quark loop. 
Thus, for the SM to be valid up to the Planck or Grand Unification Theory (GUT) scale, the radiative corrections are enormous. 
One needs un-natural amount of fine tuning to fit the Higgs mass at the observed value of 125GeV.
One of the most popular solutions to this problem is Supersymmetry (SUSY), which assigns chirality to the Higgs particle. 
SUSY solves correction problem, neutrino masses, and provides candidate of DM. 
Unfortunately, LHC has found no significant excess over the SM background in their search for SUSY\cite{SUSY}. 
Although the non-observation of superpartners does not invalidate SUSY, it makes less attractive among the particle physics community. 
Non-observation of superpartners, particularly the stop (scalar partner of the top quark) has pushed its mass beyond 1TeV. 
This generated problem of "little hierarchy", but an alternative solution of "neutral naturalness" remains. 


In framework of neutral naturalness, top partners are not charged under the SM color group. 
Because of being colorless, its cross-section of production is much less, and the present limits on the top partner particles are well below 1TeV. 
Examples of neutral naturalness models are Twin Higgs \cite{Chacko:2005pe},
Folded SUSY \cite{Burdman:2006tz}, and Quirky Little Higgs \cite{Cai:2008au}.
Theoretical models provide the possibility of neutral LLPs, which may be produced in the proton-proton
collisions of the LHC and decay back into SM particles far from the interaction point (IP).\cite{Craig:2015pha}


Interest in neutral naturalness models has increased in recent years
after the realization that they can lead to the production of long-lived particles
detectable at the LHC \cite{Curtin:2015fna,Csaki:2015fba}.
For instance, in many of Mirror and Twin SM models, only SM Higgs boson can interact with both SM QCD and mirror QCD partners.
If the mirror QCD gluons could form scalar glueballs, the SM Higgs boson can become a portal between the SM and BSM mirror QCD scalar glueballs. 
BSM mirror QCD scalar glueballs can only decay back to SM particles via Higgs portal as well. 
Since the decay like an offshell Higgs boson, its crosssections are highly suppressed. 
At the same time, decay branching ratio to highest mass fermions will be highest following the Yukawa couplings.
Thus, they may decay into b quarks or tau leptons depending on the mirror scalar's mass.
The displaced decays of the scalars will lead to exotic signatures in the LHC, such as distant innermost tracker hit, displaced vertices, and displaced jets.
The long-lived scalar model is shown in the left-panel of Figure~\ref{fig:twinhiggs}.


Searches for LLPs decaying into final states containing jets were carried out
at the Tevatron ( $\sqrt{s}$ = 1.96~TeV) by both the CDF~\cite{Aaltonen:2011rja} and D0~\cite{Abazov:2009ik} Collaborations,
at the LHC by the ATLAS and LHCb Collaborations at $\sqrt{s}$ = 7~TeV~\cite{ATLAS:2012av,Aaij:2014nma},
by the ATLAS,CMS and LHCb Collaborations at $\sqrt{s}$ = 8~TeV~\cite{Aad:2015uaa,Aad:2015rba,PhysRevD.91.012007,Aad:2015asa,Aaij:2017mic,Aaij:2016xmb,Aaij:2015ica} and more by the
CMS Collaboration~\cite{Sirunyan:2017jdo,displacedvertices,displacedjets2016,delayedjets,emergingjets,CMS-PAS-EXO-19-021}
 and ATLAS Collaboration~\cite{Aaboud:2018iil,Aaboud:2018jbr,Aaboud:2018arf,Aaboud:2018aqj,Aaboud:2018kbe,Aaboud:2019trc,Aaboud:2019opc,Aad:2019kiz,Aad:2019pfm,Aad:2019tcc,Aad:2019xav,Aad:2019tua} at $\sqrt{s}$ = 13~TeV. 
Most recently, CMS Collaboration released a result, in which Higgs are created in association with Z vector boson ~\cite{ZHAN}, for better probe into lighter scalar mass with help of clean dilepton trigger.
To date, no search has observed evidence of BSM LLPs.

Although exclusion limit on b and d-quarks were set below 1 for analysis above, exclusion limit for $\tau$ final state has been oftenly omitted or published with value above 1.  
Displaced Jets analysis face challenges for $\tau$ final state given its hadronic and leptonic decay mode and complicated reconstruction mechanism. 
However, Leptophilic model for Twin Higgs and other Higgs models are also highly motivated ~\cite{Lepto}. Continuous neglect of $\tau$ final state limit is not only a good practice, but also overlooking an important undiscovered phase space. 
This analysis searches for Higgs Portal model with Higgs' Leptophilic nature with focus on the limit value of $\tau$ final state.
The 55GeV maximum mass is to investigate only on-shell neutral scalar particles from the Higgs. 
Minimum 7GeV is for scalar particles to create on-shell tau-lepton pairs.
Feynman diagram of the scalar particle production mechanism is depicted in figure 1.
Since these scalar particles decay from the Higgs Boson, they can only be observed in high-energy particle collider, such as the LHC. 
The analysis uses data obtained from the Compact Muon Solenoid (CMS), one of the LHCs located in CERN. 

Most CMS searches are not optimal for detecting Higgs boson decays to displaced-jets
due to the soft $p_T$ nature of its decays products -- the new scalars.
%Standard CMS searches rely on $H_T$ triggers that are highly inefficienty for this signal.
Low HT signature becomes particularly more difficult with long lived signature.
Higgs produced in association with Z vector boson analysis~\cite{ZHAN} overcame this barrier with help of dilepton trigger. 
Although ggH production mode gives the largest Higgs crosssection, it complicates the trigger strategy even further. 
This analysis exploits the $\tau$ lepton's leptonic decay mode into a soft muon by with trigger of B Parking HLT Path implemented in 2018.

Another challenge for $\tau$ lepton analysis is different decay modes of $\tau$ leptons. 
$\tau$ leptons decay hadronically and leptonically with several different sub-decay modes. 
Developing analysis strategies to optimize search for each sub-decay modes are extremely complicated, a main reason for omission or no good exclusion limit in predecessor CMS results.
To be inclusive all $\tau$ leptons' decay modes, displaced vertex serach can be more efficient than displace object (jet,muon,electrons). 
We exploit the newly developed Regions of Interest mechanism in the tracker volume. 
Regions of Interest (ROI) form displaced vertex candidates by fittng pair-wise tracks of Lost-tracks and PackedPFCandidates class in MINIAOD into a vertex. 
ROIs save all relevant track and fitted vertex qualities along with isolation informaiton.
These variables are used as input for Machine Learning (ML) algorithms, enabling a highly generic and data-scientific search method.


The rest of this note is organized as follows.
In Section~\ref{sec:samples}, we describe the datasets and Monte Carlo samples, including those of the signal model, used in the search. 
The physics objects and formation of Regions of Interest are described in Section~\ref{sec:objects}.
The trigger strategy and event selections are presented in Section~\ref{sec:selections}. 
Section~\ref{sec:estimate} describes the data driven background estimate and its validation. 
Section~\ref{sec:systs} describes the systematic uncertainties.
Finally, Section~\ref{sec:results} presents the results of the search.
We conclude with Section~\ref{sec:conclusions}.
