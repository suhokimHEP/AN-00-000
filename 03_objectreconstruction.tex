\clearpage
\section{Physics object definitions}\label{sec:objects}
%reconstruction algorithms, isloation, cleaning, IDs, particle flow

In this section, we provide the definitions of physics objects used in the analysis.
We make use of Regions of Interest, muons, taus, and jets.

\subsection{Muons}\label{sec:muons}


The analysis sources SlimmedMuons from MINIAOD to produce {\tt selectedPatMuons}.
Muons require 
Muon objects require 
\begin{itemize}
  \item $\pt$ $>$ 12 GeV to reach BPH trigger plateau
  \item $|\eta|$ $<$ 1.5 due to L1 seed $|\eta|$ cut in BPH HLT path
  \item Pass the Loose ID criterion (isLooseMuon). As described in the Muon POG~\cite{muonpog}.
\end{itemize}


The Isolation requirements on muons are discussed in Section~\ref{sec:selections}.


\subsection{Jets}\label{sec:jets}

The analysis sources SlimmedJets from MINIAOD to produce {\tt selectedJets}.
CMS reconstruct jets from calorimeter energy deposits using the
anti-$k_T$ clustering algorithm with a distance parameter of $R=0.4$~\cite{Cacciari:2008gp}.
Then, the calojets are inputed into the Particle-Flow (PF) algorithms to produce PFJets. Variables in PFJets class are then slimmed to be saved into MINIAOD files. The analysis uses these SlimmedJets for the jets' b-tagging scores and others.
Jet objects require
\begin{itemize}
  \item \pt $>$ 20 GeV
  \item $|\eta|$ $<$ 2.4
  \item 0 $\leq$ emEnergyFraction $\leq$ 0.9
  \item 0 $\leq$ energyFractionHadronic $\leq$ 0.9
  \item No selected electron or muon within $\Delta R=0.4$
\end{itemize}
The energy fraction cuts above are inspired by the recommended Run2 Tight jet-ID
cuts for particle flow jets~\cite{jetid_2016,jetid_2017,jetid_2018}.

\subsection{Taus}\label{sec:taus}

The analysis sources PAT::slimmedTaus from MINIAOD for MC and RECO::slimmedTaus for Data to produce {\tt selectedTaus}.
$\tau$ objects decay hadronically for 64\% of its decay. Hadron-Plus-Strip (HPS) algorithm enables the reconstruction of $\tau$'s hadronic decay. 
HPS uses PFJets as its starting point. 
$\tau$'s hadronic decay can be reconstructed with PFJets' charged Hadrons in HCAL and 2 $\gamma$s from $\pi^{0}$ in ECAL.  
Tau objects require
\begin{itemize}
  \item \pt $>$ 20 GeV
  \item $|\eta|$ $<$ 2.4
\end{itemize}

\subsection{Region of Interest}\label{sec:ROIs}

The complete construction procedures of Regions of Interest are detailed in the following subsections.

\begin{itemize}
  \item Good quality track selection
  \item Vertex Fitted from pair-wise tracks by V0Fitter in CMSSW
  \item Cluster the fitted vertices to form a Region of Intrest (ROI)
  \item Look for tracks around $\Delta R=0.3$ to save ROI isolation information
\end{itemize}

\subsubsection{Tracks}\label{sec:ROI_tracks}

The analysis sources packedPFCandidates and lostTracks from MINIAOD.
Track parameters and convariance values will be propagated along the ROI production process and no value should either be infinite or N/A
\begin{itemize}
  \item !isinf(tracks.parameter)  $&&$ !isnan(tracks.parameter) 
  \item !isinf(tracks.covariance) $&&$ !isnan(tracks.covariance) 
  \item Number of valid hits $>$ 3
  \item \pt $>$ 0.35
  \item Track $IPSig_{XY}>$2.
  \item Track $IPSig_{Z}>$-1.
  \item Track normalized $\chi^{2}<$10.
\end{itemize}


\subsubsection{Vertex Fitter}\label{sec:ROI_V0Fitter}

The analysis sources offlineBeamspot from MINIAOD for beamspot reference.
Vertex fitter is KalmanVertexFitter with cuts on the vertex
\begin{itemize}
  \item Vertex $\chi^{2}<$6.63 
  \item Transverse Decay distance significance$>$15.
  \item V0mass $<$13000GeV
  \item cos($\theta_{XY}$) betwwen x and p of V0 candidate $>$ 0
  \item cos($\theta_{XYZ}$) betwwen x and p of V0 candidate $>$ -2
\end{itemize}


\subsubsection{ROI formation}\label{sec:ROI_ROIformation}

Fitted V0s are clustered to form a Region of Interest (ROI).
These ROIs have cuts on their parameters as below.
\begin{itemize}
  \item Radius of ROI $<$ 1 cm
  \item Annulus $\Delta R <$ 0.3 
\end{itemize}


